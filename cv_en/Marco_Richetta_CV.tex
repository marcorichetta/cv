%% start of file `template.tex'.
%% Copyright 2006-2015 Xavier Danaux (xdanaux@gmail.com), 2020-2022 moderncv maintainers (github.com/moderncv).
%
% This work may be distributed and/or modified under the
% conditions of the LaTeX Project Public License version 1.3c,
% available at http://www.latex-project.org/lppl/.

\documentclass[12pt,a4paper,sans]{moderncv}        % possible options include font size ('10pt', '11pt' and '12pt'), paper size ('a4paper', 'letterpaper', 'a5paper', 'legalpaper', 'executivepaper' and 'landscape') and font family ('sans' and 'roman')

% moderncv themes
\moderncvstyle{classic}                            % style options are 'casual' (default), 'classic', 'banking', 'oldstyle' and 'fancy'
\moderncvcolor{blue}                               % color options 'black', 'blue' (default), 'burgundy', 'green', 'grey', 'orange', 'purple' and 'red'
%\renewcommand{\familydefault}{\sfdefault}         % to set the default font; use '\sfdefault' for the default sans serif font, '\rmdefault' for the default roman one, or any tex font name

\usepackage{amsmath} % for math

% adjust the page margins
\usepackage[scale=0.85]{geometry}
\setlength{\hintscolumnwidth}{3.8 cm}                % if you want to change the width of the column with the dates
%\setlength{\makecvheadnamewidth}{10cm}            % for the 'classic' style, if you want to force the width allocated to your name and avoid line breaks. be careful though, the length is normally calculated to avoid any overlap with your personal info; use this at your own typographical risks...

% font loading
% for luatex and xetex, do not use inputenc and fontenc
% see https://tex.stackexchange.com/a/496643
\ifxetexorluatex
  \usepackage{fontspec}
  \usepackage{unicode-math}
  \defaultfontfeatures{Ligatures=TeX}
  \setmainfont{Latin Modern Roman}
  \setsansfont{Latin Modern Sans}
  \setmonofont{Latin Modern Mono}
  \setmathfont{Latin Modern Math} 
\else
  \usepackage[T1]{fontenc}
  \usepackage{lmodern}
\fi

% document language
\usepackage[english]{babel}  % FIXME: using spanish breaks moderncv

% personal data
\name{Marco Richetta}{}
% \familyname{}
\address{Copenhagen, Denmark}{}
\email{marcorichetta@gmail.com}
\homepage{marcorichetta.vercel.app}

\social[linkedin]{marco-richetta}
\social[github]{marcorichetta}
% Social icons
% \social[linkedin]{john.doe}                        % optional, remove / comment the line if not wanted
% \social[xing]{john\_doe}                           % optional, remove / comment the line if not wanted
% \social[twitter]{ji\_doe}                          % optional, remove / comment the line if not wanted
% \social[github]{jdoe}                              % optional, remove / comment the line if not wanted
% \social[gitlab]{jdoe}                              % optional, remove / comment the line if not wanted
% \social[stackoverflow]{0000000/johndoe}            % optional, remove / comment the line if not wanted
% \social[bitbucket]{jdoe}                           % optional, remove / comment the line if not wanted
% \social[skype]{jdoe}                               % optional, remove / comment the line if not wanted
% \social[orcid]{0000-0000-000-000}                  % optional, remove / comment the line if not wanted
% \social[researchgate]{jdoe}                        % optional, remove / comment the line if not wanted
% \social[researcherid]{jdoe}                        % optional, remove / comment the line if not wanted
% \social[telegram]{jdoe}                            % optional, remove / comment the line if not wanted
% \social[whatsapp]{12345678901}                     % optional, remove / comment the line if not wanted
% \social[signal]{12345678901}                       % optional, remove / comment the line if not wanted
% \social[matrix]{@johndoe:matrix.org}               % optional, remove / comment the line if not wanted
% \social[googlescholar]{googlescholarid}            % optional, remove / comment the line if not wanted

% new command for cventry (this is done to allow users unbold or unitalicize the text in the cventry command)
\renewcommand*{\cventry}[6][.25em]{%
  \cvitem[#1]{#2}{%
    #3%
    \ifthenelse{\equal{#4}{}}{}{, #4}%
    \ifthenelse{\equal{#5}{}}{}{, #5}%
    \ifthenelse{\equal{#6}{}}{}{, #6}%
  }
}



\begin{document}
    \maketitle


    % save the original href command in a new command:
    \let\hrefWithoutArrow\href
     % new command for external links:
    \renewcommand{\href}[2]{\hrefWithoutArrow{#1}{\mbox{\color{color1} #2 \raisebox{.15ex}{\footnotesize \faExternalLink*}}}}

    \hypersetup{pdftitle={Marco Richetta's CV}}

    \section{About Me}

        \cvline{}{I like to share knowledge by \href{https://www.youtube.com/live/EWofHnNngoc?si=1OpnvzkN-R0NSc2i}{giving} \href{https://www.linkedin.com/in/marco-richetta/overlay/experience/2020463733/multiple-media-viewer/?profileId=ACoAABQf7hUBZUJgMc0bcZABQZwPRzETisnwoTM&treasuryMediaId=1635515512870}{talks}, also as a \href{https://frontend.cafe/mentorias}{mentor} or simply helping others to \href{https://forum.djangoproject.com/u/marcorichetta/summary}{solve problems}.}

        \cvline{}{Currently interested in enhancing \textbf{Developer Experience (DX)} through IaC, MLOps, Documentation and Testing. Also interested in Go, Functional Programming.}


    
    \section{Experience}

        \cventry{July 2022 – June 2024}{\textbf{Mercado Libre}}{Software Engineer}{Córdoba, Argentina}{}{}
        \cvlistitem{Worked improving \textbf{Python DX}, maintaining tooling used by 1000s of developers in the company. (General support, features, bugfixes, version deprecation).}
        \cvlistitem{Building on our Python expertise, we led upgrades for over 1,000 outdated Java applications and libraries using \href{https://sourcegraph.com/batch-changes}{Large-Scale Code Changes}, managing the roadmap and collaborating with responsible teams to implement and monitor each update.}


        \cventry{Jan 2021 – June 2022}{\textbf{Flux IT}}{Software Developer}{Córdoba, Argentina}{}{}
        \cvlistitem{Fullstack development to integrate 3rd party systems with internal company applications.}
        \cvlistitem{Later, in a more DevOps role, I helped in the migration of the internal infrastructure to Kubernetes applying GitOps. This also led to better DX.}



    
    \section{Education}

        \cventry{Dec 2020}{\textbf{Instituto Superior 25 de Mayo}}{Computer Systems Analyst}{}{}{}



    
    \section{Technologies}

        \cvline{Languages}{Python, React/JavaScript, Go}

        \cvline{Technologies}{Django, Pytest, Kubernetes, FluxCD, Docker}


    

\end{document}